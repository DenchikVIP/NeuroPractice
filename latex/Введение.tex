\section*{ВВЕДЕНИЕ}
\addcontentsline{toc}{section}{ВВЕДЕНИЕ}

В условиях активного развития информационных технологий и увеличения объемов данных, особенно актуальной становится задача автоматизации процессов анализа визуальной информации. Прогресс в области машинного зрения и искусственного интеллекта позволяет создавать системы, способные эффективно распознавать и классифицировать объекты на изображениях, полученных с беспилотных летательных аппаратов (БПЛА). Использование глубоких нейронных сетей и алгоритмов компьютерного зрения открывает новые возможности для мониторинга и реагирования на чрезвычайные ситуации, такие как возгорания.

Современные исследования в области искусственного интеллекта направлены на создание алгоритмов, способных анализировать сложные и разнообразные данные с высокой степенью точности. Разработка интеллектуальной системы распознавания и классификации возгораний является одним из таких направлений. Эта система предназначена для оперативного обнаружения и точной классификации возгораний, что может значительно повысить эффективность принятия решений при борьбе с огненными стихиями.

\emph{Целью данной работы} является разработка интеллектуальной системы, использующей данные с БПЛА для распознавания и классификации возгораний. Для достижения цели были поставлены и решены \emph{следующие задачи}:
\begin{itemize}
\item исследовать существующие методы и подходы в области машинного зрения для распознавания возгораний;
\item разработать алгоритмы для обработки и анализа изображений, полученных с БПЛА;
\item создать модель глубокой нейронной сети для классификации типов возгораний;
\item провести экспериментальное тестирование разработанной системы.
\end{itemize}

\emph{Структура и объем работы.} Отчет состоит из введения, 4 разделов основной части, заключения, списка использованных источников, 2 приложений. Текст выпускной квалификационной работы равен \formbytotal{page}{страниц}{е}{ам}{ам}.

\emph{Во введении} сформулирована цель работы, поставлены задачи разработки, описана структура работы, приведено краткое содержание каждого из разделов.

\emph{В первом разделе} на стадии описания технической характеристики предметной области приводится сбор информации и файлов для включающей выборки при обучении, на которой осуществляется обучение нейронной сети.

\emph{Во втором разделе} на стадии технического задания приводятся требования к разрабатываемому приложению.

\emph{В третьем разделе} на стадии технического проектирования представлены проектные решения для приложения по классификации возгораний.

\emph{В четвертом разделе} приводится список классов и их методов, использованных при разработке приложения, производится тестирование разработанной интеллектуальной системы.

В заключении излагаются основные результаты работы, полученные в ходе разработки.

В приложении А представлен графический материал.
В приложении Б представлены фрагменты исходного кода. 
