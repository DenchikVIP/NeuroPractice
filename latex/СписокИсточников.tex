\addcontentsline{toc}{section}{СПИСОК ИСПОЛЬЗОВАННЫХ ИСТОЧНИКОВ}

\begin{thebibliography}{50}

    \bibitem{neuralnetworks} Хайкин, С. Нейронные сети: полный курс, 2-е издание / С. Хайкин. – Москва : Вильямс, 2018. – 1104 с. – ISBN 978-5-8459-2101-0. – Текст~: непосредственный.
    \bibitem{python} Лутц, М. Изучаем Python, 5-е издание / М. Лутц. – Санкт-Петербург : Питер, 2019. – 1584 с. – ISBN 978-5-4461-0705-9. – Текст~: непосредственный.
    \bibitem{deeplearning} Гудфеллоу, И., Бенджио, Ю., Курвилль, А. Глубокое обучение / И. Гудфеллоу, Ю. Бенджио, А. Курвилль. – Москва : ДМК Пресс, 2017. – 652 с. – ISBN 978-5-97060-487-9. – Текст~: непосредственный.
    \bibitem{machinelearning} Мерфи, К. Машинное обучение: вероятностный подход / К. Мерфи. – Москва : ДМК Пресс, 2018. – 704 с. – ISBN 978-5-97060-212-7. – Текст~: непосредственный.
    \bibitem{pythonmachinelearning} Рашка, С., Мирджалили, В. Python и машинное обучение / С. Рашка, В. Мирджалили. – Москва : ДМК Пресс, 2018. – 418 с. – ISBN 978-5-97060-310-0. – Текст~: непосредственный.
    \bibitem{tensorflow} Жолковский, Е. К. TensorFlow для профессионалов / Е. К. Жолковский. – Москва : ДМК Пресс, 2019. – 480 с. – ISBN 978-5-97060-746-7. – Текст~: непосредственный.
    \bibitem{keras} Чоллет, Ф. Глубокое обучение на Python / Ф. Чоллет. – Москва : ДМК Пресс, 2018. – 304 с. – ISBN 978-5-97060-409-1. – Текст~: непосредственный.
    \bibitem{fuzzysystems} Клейн, Р. Нечеткие системы в Python / Р. Клейн. – Москва : ДМК Пресс, 2020. – 320 с. – ISBN 978-5-97060-758-0. – Текст~: непосредственный.
    \bibitem{advancedpython} Бейдер, Д. Python Tricks: A Buffet of Awesome Python Features / Д. Бейдер. – Москва : ДМК Пресс, 2021. – 300 с. – ISBN 978-5-97060-999-7. – Текст~: непосредственный.
    \bibitem{practicalml} Герон, О. Практическое машинное обучение с Scikit-Learn и TensorFlow / О. Герон. – Москва : ДМК Пресс, 2019. – 572 с. – ISBN 978-5-97060-524-1. – Текст~: непосредственный.
    \bibitem{neuralnetspython} Нильсен, М. Нейронные сети и глубокое обучение / М. Нильсен. – Москва : ДМК Пресс, 2021. – 250 с. – ISBN 978-5-97060-777-1. – Текст~: непосредственный.
    \bibitem{uml} Буч, Г. Введение в UML от создателей языка / Г. Буч, И. Якобсон, Д. Рамбо. – Москва : ДМК Пресс, 2015. – 498 с. – ISBN 978-5-457-43379-3. – Текст~: непосредственный.
    \bibitem{uml2} Джеймс, Р. UML 2.0. Объектно-ориентированное моделирование и разработка / Р. Джеймс, Б. Майкл. – 2-е изд. – Санкт-Петербург : Питер, 2021. – 542 с. – ISBN 978-5-4461-9428-5. – Текст~: непосредственный.
    \bibitem{oopanalyz} Зайцев, М. Г. Объектно-ориентированный анализ и программирование / М. Г. Зайцев. – Новосибирск : изд-во НГТУ, 2017. – 84 с. – ISBN 978-5-04-112962-0. – Текст~: непосредственный.
    \bibitem{interface} Мандел, Т. Разработка пользовательского интерфейса / Т. Мандел. – ДМК Пресс, 2019. – 420 с. – ISBN 978-5-04-195060-6. – Текст~: непосредственный.
    \bibitem{intsysBPLA} Интеллектуальная система обработки изображений, получаемых с беспилотных летательных аппаратов / Томакова Р.А., Филист С. А., Нефедов Н. Г. [и др.]. – Текст : непосредственный // Известия Юго-Западного государственного университета. Серия: Управление, вычислительная техника, информатика. Медицинское приборостроение. 2022. Т. 12(4): № 4. С. 64-85.
    \bibitem{traektBPLA} Метод и алгоритм автономного планирования траектории полета беспилотного летательного аппарата при мониторинге пожарной обстановки в целях раннего обнаружения источника возгорания / Томакова Р.А., Филист С. А., Брежнева А. Н. [и др.] – Текст : непосредственный // Известия Юго-Западного государственного университета. Серия: Управление, вычислительная техника, информатика. Медицинское приборостроение. 2023. Т. 13(1): № 1. С. 93-111.
    \bibitem{monitoring} Информационная система мониторинга на основе интеллектуальной классификации изображений видеопотоков / Томакова Р.А. ,Брежнев А.В., Брежнева А.Н. -  Текст : непосредственный // Информационное общество. .2023. №5. С. 134-143.
    \bibitem{Cos} Томакова Р.А.  Методы и алгоритмы цифровой обработки изображений : учебное пособие / Р. А. Томакова, Е. А. Петрик ; Юго-Зап. гос. ун-т. - Курск : Университетская книга, 2020. - 310 с. - Библиогр.: с. 297-309. - ISBN 978-5-907270-19-0. - Текст : непосредственный.
    \bibitem{teoriaNS} Томакова Р.А.  Основы теории нейрокомпьютерных систем : учебное пособие / Р. А. Томакова ; Юго-Зап. гос. ун-т. - Курск : [б. и.], 2021. - 135 с. - ISBN 978-5-907555-36-5 : Б. ц. - Текст : непосредственный.
    \bibitem{methodbase} Томакова Р.А.  Методологические основы научных исследований : учебное пособие / Р. А. Томакова, В. И. Томаков ; Юго-Зап. гос. ун-т. - Курск : ЮЗГУ, 2017. - 204 с. - Библиогр.: с. 199-203. - ISBN 978-5-7681-1210-3. - Текст : непосредственный.
    \bibitem{py1} Чаплыгин А.А. Программирование на языке Python : учебное пособие / Юго-Зап. гос. ун-т ; сост. А. А. Чаплыгин. - Электрон. текстовые дан. (229 КБ). - Курск : ЮЗГУ, 2021. - 15 с. - Загл. с титул. экрана. - Б. ц. - Текст : электронный.
    \bibitem{py2} Северенс, Ч.   Введение в программирование на Python : учебник / Ч. Северенс. - 2-е изд., испр. - Москва : Национальный Открытый Университет «ИНТУИТ», 2016. - 231 с. - URL: \url{https://biblioclub.ru/index.php?page=book&id=429184} (дата обращения: 24.08.2023) . - Б. ц. - Текст : электронный.
    \bibitem{py3} Хахаев, И. А.   Практикум по алгоритмизации и программированию на Python: курс : учебное пособие / И. А. Хахаев. - 2-е изд., исправ. - Москва : Национальный Открытый Университет «ИНТУИТ», 2016. - 179 с. - URL: \url{https://biblioclub.ru/index.php?page=book&id=429256} (дата обращения: 24.08.2023) . - Библиогр. в кн. - Б. ц. - Текст : электронный.
    \bibitem{statanaliz} Программные системы статистического анализа: обнаружение закономерностей в данных с использованием системы R и языка Python : учебное пособие / В. М. Волкова, М. А. Семенова, Е. С. Четвертакова, С. С. Вожов. - Новосибирск : Новосибирский государственный технический университет, 2017. - 74 с. : ил., табл. - URL: \url{https://biblioclub.ru/index.php?page=book&id=576496} (дата обращения: 02.03.2022) . - Режим доступа: по подписке. - Библиогр.: с. 48. - ISBN 978-5-7782-3183-2 : Б. ц. - Текст : электронный.
    \bibitem{pyyavu} Шелудько, В. М.  Язык программирования высокого уровня Python: функции, структуры данных, дополнительные модули : учебное пособие / В. М. Шелудько ; Министерство науки и высшего образования РФ ; Федеральное государственное автономное образовательное учреждение высшего образования «Южный федеральный университет» ; Институт компьютерных технологий и информационной безопасности. - Ростов-на-Дону|Таганрог : Издательство Южного федерального университета, 2017. - 108 с. : ил. - URL: \url{http://biblioclub.ru/index.php?page=book&id=500060} (дата обращения: 24.08.2023) . - Режим доступа: по подписке. - Библиогр. в кн. - ISBN 978-5-9275-2648-2 : Б. ц. - Текст : электронный.
    \bibitem{nechetk} Емельянов С.Г.   Интеллектуальные системы на основе нечеткой логики и мягких арифметических операций : учебник / С. Г. Емельянов, В. С. Титов, М. В. Бобырь. - Москва : Аргамак-Медиа, 2014. - 338, [7] с. : табл., граф. - Библиогр.: с. 325-336. - 300 экз. - ISBN 978-5-00024-035-9 –  Текст : непосредственный.
    \bibitem{prinreshen} Демидова Л.А.  Принятие решений в условиях неопределенности : монография / Л. А. Демидова. - 2-е изд., перераб. - Москва : Горячая линия - Телеком, 2016. - 289 с. - ISBN 978-5-9912-0513-9 : 368.68 р. - Текст : непосредственный.
    \bibitem{teorians2} Яхъяева, Г. Э.   Основы теории нейронных сетей : [Электронный ресурс] / Г. Э. Яхъяева. - 2-е изд., испр. - Москва : Национальный Открытый Университет «ИНТУИТ», 2016. - 200 с. - (Основы информационных технологий). - URL: \url{http://biblioclub.ru/index.php?page=book&id=429110}. - ISBN 978-5-94774-818-5 : Б. ц. - Текст : электронный.
    \bibitem{dinamstruct} Лубенцова, Е. В.    Системы управления с динамическим выбором структуры, нечеткой логикой и нейросетевыми моделями : монография / Е. В. Лубенцова. - Ставрополь : СКФУ, 2014. - 248 с. - URL: \url{http://biblioclub.ru/index.php?page=book&id=457413} (дата обращения: 28.04.2022) . - Режим доступа: по подписке. - ISBN 978-5-88648-902-6 : Б. ц. - Текст : электронный.
    \bibitem{matteoria}Гелиг, А. Х.    Введение в математическую теорию обучаемых распознающих систем и нейтронных сетей : [Электронный ресурс] : учебное пособие / А. Х. Гелиг, А. С. Матвеев. - Санкт-Петербург : Издательство Санкт-Петербургского Государственного Университета, 2014. - 224 с. - (Прикладная математика и информатика). - URL: \url{http://biblioclub.ru/index.php?page=book&id=457945}. - ISBN 978-5-288-05551-5 : Б. ц. - Текст : электронный.
    \bibitem{naukaodata} Келлехер, Д.    Наука о данных: базовый курс : учебное пособие / Д. Келлехер, Б. Тирни ; науч. ред. З. Мамедьяров ; пер. с англ. М. Белоголовский. - Москва : Альпина Паблишер, 2020. - 224 с. - URL: \url{http://biblioclub.ru/index.php?page=book&id=598235} (дата обращения: 09.09.2022) . - Режим доступа: по подписке. - ISBN 978-5-9614-3170-4 : Б. ц. - Текст : электронный.
    \bibitem{matlab} Кассим К.Д.А.   Моделирование систем искусственного интеллекта в среде Matlab и Fuzzytech : учебное пособие / К. Д. А. Кассим, С. А. Филист, О. В. Шаталова. - Курск : Деловая полиграфия, 2016. - 186 с. - Библиогр.: с. 185. - ISBN 978-5-9908582-8-2. - Текст : непосредственный.
    \bibitem{serebrovsk} Математические методы и инновационные научно-технические разработки : сборник научных трудов / Федеральное государственное бюджетное образовательное учреждение высшего профессионального образования "Юго-Западный государственный университет" ; редкол.: В. В. Серебровский (отв. ред.) [и др.]. - Курск : ЮЗГУ, 2014. - 282 с. ; 20. - Библиогр. в конце ст. - 100 экз. - ISBN 978-5-7681-0930-1 : 380.00 р. - Текст : непосредственный.
    \bibitem{planirovanieobj} Интеллектуальное планирование траекторий подвижных объектов в средах с препятствиями / Д. А. Белоглазов [и др.] ; под ред. В. Х. Пшихопова. - Москва : Физматлит, 2014. - 295 с. : ил. - Авт. указ. на обороте тит. л. - Библиогр.: с. 276-295 (314 назв.). - ISBN 978-5-9221-1595-7. - Текст : непосредственный.
    \bibitem{autoref} Волков Д.А.  Модель, метод и нейросетевое оптико-электронное вычислительное устройство распознавания изображений : специальность 05.13.05 "Элементы и устройства вычислительной техники и систем управления" : автореферат диссертации на соискание ученой степени кандидата технических наук / Волков Денис Андреевич ; Юго-Западный государственный университет. - Курск, 2020. - 17 с. - Место защиты: ФГБОУ ВО "Юго-Западный государственный университет" (Курск). - Текст : непосредственный.
    \bibitem{allfill} Программирование, тестирование, проектирование, нейросети, технологии аппаратно‐программных средств (практические задания и способы их решения) : учебник / С. В. Веретехина, К. С. Кармицкий, Д. Д. Лукашин [и др.]. - Москва : Директ-Медиа, 2022. - 144 с. - URL: \url{https://biblioclub.ru/index.php?page=book&id=694782} (дата обращения: 10.01.2023) . - Режим доступа: по подписке. - Библиогр. в кн. - ISBN 978-5-4499-3321-8 : Б. ц. - Текст : электронный.
    \bibitem{intsysandtech} Интеллектуальные системы и технологии : учебное пособие / С. П. Ющенко [и др.] ; Юго-Зап. гос. ун-т. - Курск : Университетская книга, 2018. - 226 с. : ил. - Библиогр.: с. 238-240 (32 назв.). - ISBN 978-5-907138-22-3 : 560.00 р. - Текст : непосредственный.
    \bibitem{} Системная инженерия. Принципы и практика = Systems engineering principles and practice : учебник / А. Косяков [и др.] ; пер. с англ. под ред. В. К. Батоврин. - 2-е изд. - Москва : ДМК Пресс, 2014. - 624 с. : ил. - Указ.: с. 610-619. - 400 экз. - ISBN 978-5-97060-122-8 (в пер.). - Текст : непосредственный.
    \bibitem{} Сидоркина И.Г..   Системы искусственного интеллекта : учебное пособие / И. Г. Сидоркина. - Москва : КНОРУС, 2016. - 246 с. : рис. - Библиогр.: с. 244-245. - ISBN 978-5-406-04876-4 . - Текст : непосредственный.
    \bibitem{} Бабенко Л.К.   Параллельные алгоритмы для решения задач защиты информации : монография / Л. К. Бабенко, Е. А. Ищукова, И. Д. Сидоров. - Москва : Горячая линия-Телеком, 2014. - 304 с. : ил. - Библиогр.: с. 222-224. - ISBN 978-5-9912-0426-2 : 340.00 р. - Текст : непосредственный.
    \bibitem{} Кузнецов, А. С.    Теория вычислительных процессов : [ Электронный ресурс] : учебник / А. С. Кузнецов, Р. Ю. Царев, А. Н. Князьков. - Красноярск : Сибирский федеральный университет, 2015. - 184 с. - URL: \url{http://biblioclub.ru/index.php?page=book&id=435696}. - ISBN 978-5-7638-3193-1 : Б. ц.  - Текст : электронный.
    \bibitem{} Исакова, А. И.    Основы информационных технологий : учебное пособие / А. И. Исакова. - Томск : ТУСУР, 2016. - 206 с. : ил. - URL: \url{http://biblioclub.ru/index.php?page=book&id=480808} (дата обращения: 18.02.2022) . - Режим доступа: по подписке. - Библиогр.: с. 197-198. - Б. ц. - Текст : электронный.
    \bibitem{} Гунько, А. В.    Программирование (в среде Windows) : учебное пособие / А. В. Гунько ; Новосибирский государственный технический университет. - Новосибирск : Новосибирский государственный технический университет, 2019. - 155 с. - URL: \url{https://biblioclub.ru/index.php?page=book&id=575417} (дата обращения: 03.05.2024) . - Режим доступа: по подписке. - Библиогр. в кн. - ISBN 978-5-7782-3890-9 : Б. ц. - Текст : электронный.
    \bibitem{} Малоразмерные беспилотные летательные аппараты: задачи обнаружения и пути их решения : монография / И. И. Олейник, А. А. Черноморец, В. Г. Андронов [и др.] ; под ред. В. Г. Андронова ; Юго-Зап. гос. ун-т. - Курск : ЮЗГУ, 2021. - 171 с. - Библиогр.: с. 149-170. - ISBN 978-5-7681-1518-0. - Текст : непосредственный.
    \bibitem{} Методологические основы обнаружения малоразмерных беспилотных летательных аппаратов на основе комплексной субполосной обработки сверхкороткоимпульсных радиолокационных и оптических сигналов : монография / И. И. Олейник, А. А. Черноморец, Д. С. Коптев [и др.] ; под общ. ред. В. Г. Андронов ; Юго-Зап. гос. ун-т. - Курск : ЮЗГУ, 2021. - 204 с. - Библиогр.: с. 198-203. - ISBN 978-5-7681-1526-5 . - Текст : непосредственный.
    \bibitem{} Шевцов, Максим Викторович.
    Система мониторинга пожарной и медико-экологической безопасности с использованием анализа видеоданных с беспилотных летательных аппаратов : специальность 2.3.1 "Системный анализ, управление и обработка информации (технические науки)" : автореферат диссертации на соискание ученой степени кандидата технических наук / Шевцов Максим Викторович ; Академия Государственной противопожарной службы МЧС России (Москва). - Электрон. текстовые дан. (476 КБ). - Москва, 2022. - 20 с. - Место защиты: Юго-Западный государственный университет (Курск). - Текст : непосредственный.
    \bibitem{} Аверченков, В. И.    Основы математического моделирования технических систем : учебное пособие / В. И. Аверченков, В. П. Федоров, М. Л. Хейфец. - 4-е изд., стер. - Москва : Флинта, 2021. - 271 с. - URL: \url{http://biblioclub.ru/index.php?page=book&id=93344} (дата обращения: 16.02.2024) . - Режим доступа: по подписке. - ISBN 978-5-9765-1278-8 : Б. ц. - Текст : электронный.
    \bibitem{} Зыков, С. В.    Введение в теорию программирования. Объектно-ориентированный подход : курс лекций / С. В. Зыков. - 2-е изд., испр. - Москва : Национальный Открытый Университет «ИНТУИТ», 2016. - 189 с. - (Основы информационных технологий). - URL:\url{ http://biblioclub.ru/index.php?page=book&id=429073} (дата обращения: 13.05.2024) . - Режим доступа: по подписке. - ISBN 5-9556-0009-4 : Б. ц. - Текст : электронный.
    \bibitem{} Сазонов C.Ю.    Системный подход к моделированию процессов возникновения и развития пожаров : монография / С. Ю. Сазонов ; Юго-Зап. гос. ун-т. - Курск : Деловая полиграфия, 2016. - 218 с. - Библиогр.: с. 213-219. - ISBN 978-5-9907910-9-1. - Текст : непосредственный.
\end{thebibliography}
