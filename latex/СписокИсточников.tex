\addcontentsline{toc}{section}{СПИСОК ИСПОЛЬЗОВАННЫХ ИСТОЧНИКОВ}

\begin{thebibliography}{15}

    \bibitem{neuralnetworks} Хайкин, С. Нейронные сети: полный курс, 2-е издание / С. Хайкин. – Москва : Вильямс, 2018. – 1104 с. – ISBN 978-5-8459-2101-0. – Текст~: непосредственный.
    \bibitem{python} Лутц, М. Изучаем Python, 5-е издание / М. Лутц. – Санкт-Петербург : Питер, 2019. – 1584 с. – ISBN 978-5-4461-0705-9. – Текст~: непосредственный.
    \bibitem{deeplearning} Гудфеллоу, И., Бенджио, Ю., Курвилль, А. Глубокое обучение / И. Гудфеллоу, Ю. Бенджио, А. Курвилль. – Москва : ДМК Пресс, 2017. – 652 с. – ISBN 978-5-97060-487-9. – Текст~: непосредственный.
    \bibitem{machinelearning} Мерфи, К. Машинное обучение: вероятностный подход / К. Мерфи. – Москва : ДМК Пресс, 2018. – 704 с. – ISBN 978-5-97060-212-7. – Текст~: непосредственный.
    \bibitem{pythonmachinelearning} Рашка, С., Мирджалили, В. Python и машинное обучение / С. Рашка, В. Мирджалили. – Москва : ДМК Пресс, 2018. – 418 с. – ISBN 978-5-97060-310-0. – Текст~: непосредственный.
    \bibitem{tensorflow} Жолковский, Е. К. TensorFlow для профессионалов / Е. К. Жолковский. – Москва : ДМК Пресс, 2019. – 480 с. – ISBN 978-5-97060-746-7. – Текст~: непосредственный.
    \bibitem{keras} Чоллет, Ф. Глубокое обучение на Python / Ф. Чоллет. – Москва : ДМК Пресс, 2018. – 304 с. – ISBN 978-5-97060-409-1. – Текст~: непосредственный.
    \bibitem{fuzzysystems} Клейн, Р. Нечеткие системы в Python / Р. Клейн. – Москва : ДМК Пресс, 2020. – 320 с. – ISBN 978-5-97060-758-0. – Текст~: непосредственный.
    \bibitem{advancedpython} Бейдер, Д. Python Tricks: A Buffet of Awesome Python Features / Д. Бейдер. – Москва : ДМК Пресс, 2021. – 300 с. – ISBN 978-5-97060-999-7. – Текст~: непосредственный.
    \bibitem{practicalml} Герон, О. Практическое машинное обучение с Scikit-Learn и TensorFlow / О. Герон. – Москва : ДМК Пресс, 2019. – 572 с. – ISBN 978-5-97060-524-1. – Текст~: непосредственный.
    \bibitem{neuralnetspython} Нильсен, М. Нейронные сети и глубокое обучение / М. Нильсен. – Москва : ДМК Пресс, 2021. – 250 с. – ISBN 978-5-97060-777-1. – Текст~: непосредственный.
    \bibitem{uml} Буч, Г. Введение в UML от создателей языка / Г. Буч, И. Якобсон, Д. Рамбо. – Москва : ДМК Пресс, 2015. – 498 с. – ISBN 978-5-457-43379-3. – Текст~: непосредственный.
    \bibitem{uml2} Джеймс, Р. UML 2.0. Объектно-ориентированное моделирование и разработка / Р. Джеймс, Б. Майкл. – 2-е изд. – Санкт-Петербург : Питер, 2021. – 542 с. – ISBN 978-5-4461-9428-5. – Текст~: непосредственный.
    \bibitem{oopanalyz} Зайцев, М. Г. Объектно-ориентированный анализ и программирование / М. Г. Зайцев. – Новосибирск : изд-во НГТУ, 2017. – 84 с. – ISBN 978-5-04-112962-0. – Текст~: непосредственный.
    \bibitem{interface} Мандел, Т. Разработка пользовательского интерфейса / Т. Мандел. – ДМК Пресс, 2019. – 420 с. – ISBN 978-5-04-195060-6. – Текст~: непосредственный.
\end{thebibliography}
