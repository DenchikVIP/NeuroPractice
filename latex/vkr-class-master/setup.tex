\documentclass{vkr}
\usepackage[english, russian]{babel} % переносы
\usepackage{graphicx} % для вставки картинок
\graphicspath{{images/}} % путь к изображениям
\usepackage[hidelinks]{hyperref}
\usepackage{float} % определяет метод H для рисунка с переносом на следующую страницу, ели не помещается
\usepackage{pdflscape}
\addto{\captionsrussian}{\renewcommand{\refname}{СПИСОК ИСПОЛЬЗОВАННЫХ ИСТОЧНИКОВ}}
\usepackage{xltabular} % для вставки таблиц
\usepackage{makecell}
\renewcommand\theadfont{} % шрифт в /thead
\usepackage{array} % для определения новых типов столбцов таблиц
\newcolumntype{T}{>{\centering\arraybackslash}X} % новый тип столбца T - автоматическая ширина столбца с выравниванием по центру
\newcolumntype{R}{>{\raggedleft\arraybackslash}X} % новый тип столбца R - автоматическая ширина столбца с выравниванием по правому краю
\newcolumntype{C}[1]{>{\centering\let\newline\\\arraybackslash\hspace{0pt}}m{#1}} % новый тип столбца C - фиксированная ширина столбца с выравниванием по центру
\newcolumntype{r}[1]{>{\raggedleft\arraybackslash}p{#1}} % новый тип столбца r - фиксированная ширина столбца с выравниванием по правому краю
\newcommand{\centrow}{\centering\arraybackslash} % командой \centrow можно центрировать одну ячейку (заголовок) в столбце типа X или p, оставив в оcтальных ячейках другой тип выравнивания
\newcommand{\finishhead}{\endhead\hline\endlastfoot}
\newcommand{\continuecaption}[1]{\captionsetup{labelformat=empty} \caption[]{#1}\\ \hline }
\usepackage{etoolbox}
\AtBeginEnvironment{xltabular}{\refstepcounter{tablecnt}} % подсчет таблиц xltabular, обычные таблицы подсчитываются в классе

\usepackage[tableposition=top]{caption} % подпись таблицы вверху
\captionsetup{strut=off}
\setlength{\intextsep}{0pt} % Vertical space above & below [h] floats
\setlength{\textfloatsep}{0pt} % Vertical space below (above) [t] ([b]) floats
\DeclareCaptionLabelFormat{gostfigure}{Рисунок #2} %подпись рисунка
\DeclareCaptionLabelFormat{gosttable}{Таблица #2} %подпись таблицы
\DeclareCaptionLabelSeparator{gost}{~--~} %разделитель в рисунках и таблицах
\captionsetup{labelsep=gost}
\captionsetup[figure]{aboveskip=10pt,belowskip=4mm,justification=centering,labelformat=gostfigure} % настройка подписи рисунка
\captionsetup[table]{font={stretch=1.41},skip=0pt,belowskip=0pt,aboveskip=8.5pt,singlelinecheck=off,labelformat=gosttable} % настройка подписи таблицы

\setlength{\LTpre}{8mm} % отступ сверху таблицы
\setlength{\LTpost}{6mm} % отступ снизу таблицы

\usepackage{enumitem}
\setlist{nolistsep,wide=\parindent,itemindent=*} % отступы вокруг списков, выравнивание с учетом разделителя

\usepackage{color} %% это для отображения цвета в коде
\usepackage{listings} %% листинги кода
\setmonofont[Scale=0.7]{Verdana} % моноширный шрифт для листинга

\definecolor{codegreen}{rgb}{0,0.6,0}
\definecolor{codegray}{rgb}{0.5,0.5,0.5}
\definecolor{codepurple}{rgb}{0.58,0,0.82}

\lstset{ %
language=C,                 % выбор языка для подсветки (здесь это С)
numbers=left,               % где поставить нумерацию строк (слева\справа)
numberstyle=\tiny,           % размер шрифта для номеров строк
stepnumber=1,                   % размер шага между двумя номерами строк
numbersep=5pt,                % как далеко отстоят номера строк от подсвечиваемого кода
commentstyle=\color{codegreen},
keywordstyle=\color{magenta},
numberstyle=\tiny\color{codegray},
stringstyle=\color{codepurple},
basicstyle=\linespread{0.95}\ttfamily,
backgroundcolor=\color{white}, % цвет фона подсветки - используем \usepackage{color}
showspaces=false,            % показывать или нет пробелы специальными отступами
showstringspaces=false,      % показывать или нет пробелы в строках
showtabs=false,             % показывать или нет табуляцию в строках
frame=single,              % рисовать рамку вокруг кода
tabsize=2,                 % размер табуляции по умолчанию равен 2 пробелам
captionpos=t,              % позиция заголовка вверху [t] или внизу [b] 
breaklines=true,           % автоматически переносить строки (да\нет)
breakatwhitespace=false, % переносить строки только если есть пробел
escapeinside={\%*}{*)}   % если нужно добавить комментарии в коде
}

\makeatletter % чтобы допускались русские комментарии в листингах
\lst@InputCatcodes
\def\lst@DefEC{%
 \lst@CCECUse \lst@ProcessLetter
  ^^80^^81^^82^^83^^84^^85^^86^^87^^88^^89^^8a^^8b^^8c^^8d^^8e^^8f%
  ^^90^^91^^92^^93^^94^^95^^96^^97^^98^^99^^9a^^9b^^9c^^9d^^9e^^9f%
  ^^a0^^a1^^a2^^a3^^a4^^a5^^a6^^a7^^a8^^a9^^aa^^ab^^ac^^ad^^ae^^af%
  ^^b0^^b1^^b2^^b3^^b4^^b5^^b6^^b7^^b8^^b9^^ba^^bb^^bc^^bd^^be^^bf%
  ^^c0^^c1^^c2^^c3^^c4^^c5^^c6^^c7^^c8^^c9^^ca^^cb^^cc^^cd^^ce^^cf%
  ^^d0^^d1^^d2^^d3^^d4^^d5^^d6^^d7^^d8^^d9^^da^^db^^dc^^dd^^de^^df%
  ^^e0^^e1^^e2^^e3^^e4^^e5^^e6^^e7^^e8^^e9^^ea^^eb^^ec^^ed^^ee^^ef%
  ^^f0^^f1^^f2^^f3^^f4^^f5^^f6^^f7^^f8^^f9^^fa^^fb^^fc^^fd^^fe^^ff%
  ^^^^20ac^^^^0153^^^^0152%
  % Basic Cyrillic alphabet coverage
  ^^^^0410^^^^0411^^^^0412^^^^0413^^^^0414^^^^0415^^^^0416^^^^0417%
  ^^^^0418^^^^0419^^^^041a^^^^041b^^^^041c^^^^041d^^^^041e^^^^041f%
  ^^^^0420^^^^0421^^^^0422^^^^0423^^^^0424^^^^0425^^^^0426^^^^0427%
  ^^^^0428^^^^0429^^^^042a^^^^042b^^^^042c^^^^042d^^^^042e^^^^042f%
  ^^^^0430^^^^0431^^^^0432^^^^0433^^^^0434^^^^0435^^^^0436^^^^0437%
  ^^^^0438^^^^0439^^^^043a^^^^043b^^^^043c^^^^043d^^^^043e^^^^043f%
  ^^^^0440^^^^0441^^^^0442^^^^0443^^^^0444^^^^0445^^^^0446^^^^0447%
  ^^^^0448^^^^0449^^^^044a^^^^044b^^^^044c^^^^044d^^^^044e^^^^044f%
  ^^^^0401^^^^0451%
  %%%
  ^^00}
\lst@RestoreCatcodes
\makeatother
