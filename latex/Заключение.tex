\section*{ЗАКЛЮЧЕНИЕ}
\addcontentsline{toc}{section}{ЗАКЛЮЧЕНИЕ}

Преимущества разработки интеллектуальных систем, таких как система распознавания и классификации возгораний с БПЛА, заключаются в повышении точности и скорости обработки данных. Основным ограничением является сложность обработки больших объемов информации в реальном времени.

Компании, стремящиеся к инновациям, активно внедряют передовые технологии для повышения эффективности своей деятельности. Разработка мобильного приложения позволяет оперативно реагировать на возгорания, обнаруженные с помощью БПЛА, и предоставлять данные для принятия решений.

Основные результаты работы:

\begin{enumerate}
\item Проведен анализ предметной области. Выявлены ключевые требования к системе распознавания и классификации возгораний.
\item Разработана концептуальная модель приложения. Создана модель данных для обработки и анализа изображений с БПЛА.
\item Осуществлено проектирование архитектуры приложения. Разработан пользовательский интерфейс для взаимодействия с данными возгораний.
\item Реализовано и протестировано приложение. Проведено модульное и интеграционное тестирование.
\end{enumerate}

Все заявленные требования были удовлетворены, все цели, поставленные на начальном этапе, достигнуты.
