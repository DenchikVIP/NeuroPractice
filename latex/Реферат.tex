\abstract{РЕФЕРАТ}

Объем работы равен \formbytotal{lastpage}{страниц}{е}{ам}{ам}. Работа содержит \formbytotal{figurecnt}{иллюстраци}{ю}{и}{й}, \formbytotal{tablecnt}{таблиц}{у}{ы}{}, \arabic{bibcount} библиографических источников и \formbytotal{числоПлакатов}{лист}{}{а}{ов} графического материала. Количество приложений – 2. Графический материал представлен в приложении А. Фрагменты исходного кода представлены в приложении Б.

Перечень ключевых слов: БПЛА, система машинного зрения, TensorFlow, TFRecord, распознавание образов, классификация возгораний, обработка изображений, нейронные сети, обучение с учителем, автоматизация, информационные технологии, данные с дронов, предобработка данных, моделирование, анализ данных, безопасность, аварийные ситуации, компьютерное зрение, алгоритмы, модуль, сущность, информационный блок, метод, разработчик, администратор, пользователь, приложение.

Объектом разработки является приложение, предназначенное для распознавания и классификации возгораний на изображениях, полученных с БПЛА.

Целью выпускной квалификационной работы является разработка интеллектуальной системы, предназначенной оперативно и точно идентифицировать очаги возгорания для предотвращения и минимизации ущерба от пожаров.

В процессе создания приложения были выделены основные сущности, разработаны алгоритмы для обработки и анализа изображений, использованы методы машинного обучения для обучения модели распознавания, а также создан пользовательский интерфейс для взаимодействия с системой.

При разработке приложения использовалась платформа TensorFlow для создания и обучения нейронных сетей, а также стороннее приложение LabelImg для удобной работы с изображениями.

Разработанное приложение было успешно протестировано и готово к внедрению для использования в чрезвычайных ситуациях.

\selectlanguage{english}
\abstract{ABSTRACT}
  
The volume of work is \formbytotal{lastpage}{page}{}{s}{s}. The work contains \formbytotal{figurecnt}{illustration}{}{s}{s}, \formbytotal{tablecnt}{table}{}{s}{s}, \arabic{bibcount} bibliographic sources and \formbytotal{числоПлакатов}{sheet}{}{s}{s} of graphic material. The number of applications is 2. The graphic material is presented in annex A. The layout of the site, including the connection of components, is presented in annex B.

List of keywords: UAV, computer vision system, TensorFlow, TFRecord, pattern recognition, fire classification, image processing, neural networks, supervised learning, automation, information technology, data from drones, data preprocessing, modeling, data analysis, security, emergency situations, computer vision, algorithms, module, entity, information block, method, developer, administrator, user, application.

The object of development is an application designed to recognize and classify fires in images obtained from a UAV.

The goal of the final qualifying work is to develop an intelligent system designed to quickly and accurately identify fire sources to prevent and minimize damage from fires.

In the process of creating the application, the main entities were identified, algorithms were developed for image processing and analysis, Machine learning methods were used to train the recognition model, and a user interface was created to interact with the system.

When developing the application, the TensorFlow platform was used to create and train neural networks, as well as a third-party application LabelImg for convenient work with images.

The developed application has been successfully tested and is ready for implementation for use in emergency situations.
\selectlanguage{russian}
