\section{Рабочий проект}
\subsection{Классы, используемые при разработке приложения}

Список классов и методов, которые были использованы при создании приложения представлены далее.

\subsubsection{Класс main}

Класс main явлаяется точкой входа в приложение и используется для реализации идентификации объектов с помощью нейронной сети. Здесь происходит инициализация основных компонентов программы в качестве кнопок на интерфейсе приложения. 
Описание полей и методов данного класса представлено в таблице \ref{classmain:table}.

\renewcommand{\arraystretch}{0.8} % уменьшение расстояний до сетки таблицы
\begin{xltabular}{\textwidth}{|X|p{1.5cm}|>{\setlength{\baselineskip}{0.7\baselineskip}}p{2.85cm}|>{\setlength{\baselineskip}{0.7\baselineskip}}p{4.85cm}|}
\caption{Спецификация полей класса <<main>> \label{classmain:table}}\\
\hline \centrow \setlength{\baselineskip}{0.7\baselineskip} Наименование& \centrow \setlength{\baselineskip}{0.7\baselineskip} Метод доступа & \centrow Тип данных & \centrow Описание \\
\hline \centrow 1 & \centrow 2 & \centrow 3 & \centrow 4\\ 
\hline
\endfirsthead
\caption*{Продолжение таблицы \ref{classmain:table}}\\
\hline \centrow 1 & \centrow 2 & \centrow 3 & \centrow 4\\ 
\hline
\finishhead
path\_for\_one & public & String & Хранит путь к выбранному изображению\\
\hline localizator & public & tf.keras.Model & Модель локализатора, загруженная из файла. Файл хранится в корневой папке проекта и создается в классе training. \\
\hline classifier & public & tf.keras.Model & Модель классификатора, загруженная из файла. Файл хранится в корневой папке проекта и создается в классе cllassifier. \\
\hline window & private & Tk & Главное окно приложения. Окно настроено и имеет несколько вложений соответсвтующие визуальной составляющей. Содержит в себе кнопки и поля Canvas. \\
\hline image\_field\_raw & private & Canvas & Холст для отображения исходного изображения. Содержится в окне window. \\
\hline image\_field\_ready & private & Canvas & Холст для отображения результата распознавания. Содержится в окне window. \\
\hline ConsoleRedirector & public & Class & Класс для перенаправления вывода консоли в текстовое поле text с помощью функции sys.stdout = ConsoleRedirector(text). \\
\hline image\_field\_ready & private & Canvas & Холст для отображения результата распознавания. Содержится в окне window. \\
\hline detect\_objects & private & Function & Функция для обнаружения объектов на изображении с использованием модели локализатора. Здесь происходит подготовка изображения, локализация, нормализация, разделение на элементы, нарезание изображений и сбор в массив (10, 32, 32, 3). Далее происходит классификация, счет метки класса и сбор координат в нормальный вид. \\
\hline namespace & public & Dictionary & Словарь для сопоставления индексов классов с их названиями. В нашем варианте 0 - Ничего (nothing), 1- Возгорание (fire). \\
\hline visualize & private & Function & Функция для визуализации результатов детекции с помощью OpenCV. Рисует текст и квадраты на изображении в соответствии с распознанными объектами (возгораниями) \\
\hline prettify & private & Function & Функция для улучшения результатов детекции путём объединения перекрывающихся рамок. Здесь вычисляем IoU, самый надежный способ определить совпадение. И если ни с чем не обьединили, так и оставляем результат. \\
\hline detect\_fire\_in\_image & private & Function & Функция для обнаружения огня на изображении и его визуализации. Считывает изображение, переводит его в matplotlib и выводит на Canvas поле результата. \\
\hline loadimage & public & Function & Функция для загрузки изображения через диалоговое окно. Также помещает его в Canvas поле загруженного изображения. \\
\hline detect & private & Function & Функция для запуска процесса обнаружения огня на загруженном изображении. Вызывает detect\_fire\_in\_image(). \\
\hline check\_and\_install\_packages & private & Function & Функция для проверки и установки необходимых пакетов PyQt5 и sip. Вызывается для предварительного устранения ошибок с пакетами при запуске labelimg. \\
\hline run\_labelimg & private & Function & Функция для запуска приложения LabelImg для аннотации изображений. Labelimg установлен заранее, но требует для запуска дополнительные пакеты в системе. \\
\hline create\_tfrec\_classifier & private & Function & Функция для создания TFRecord для классификатора. Вызывает функцию из другого класса. \\
\hline start\_train\_localizer & private & Function & Функция для начала обучения модели локализатора. Вызывает функцию train() и loadmodel() из другого класса. \\
\hline train\_window\_open & private & Function & Открывает новое окно для обучения нейронной сети. Здесь реализованы функции для удобного обучения, тестирования и работы с файлами нейронной сети. \\
\hline load\_model\_data & private & Function & Функция для загрузки пользовательской модели нейросети.
\end{xltabular}
\renewcommand{\arraystretch}{1.0} % восстановление сетки

\subsubsection{Класс dataprocessing}

Класс dataprocessing отвечает за обработку данных и изображений. Класс предназначен для обработки данных из XML файла, который содержит аннотации для изображений, используемых в наших задачах. В классе парсится XML файл, извлекается информацию о размеченных объектах и их координатах, загружается соответствующее изображение и визуализирует его. 
Описание полей и методов данного класса представлено в таблице \ref{classdataprocessing:table}.

\renewcommand{\arraystretch}{0.8} % уменьшение расстояний до сетки таблицы
\begin{xltabular}{\textwidth}{|X|p{2.5cm}|>{\setlength{\baselineskip}{0.7\baselineskip}}p{4.85cm}|>{\setlength{\baselineskip}{0.7\baselineskip}}p{4.85cm}|}
\caption{Спецификация полей класса <<dataprocessing>> \label{classdataprocessing:table}}\\
\hline \centrow \setlength{\baselineskip}{0.7\baselineskip} Наименование& \centrow \setlength{\baselineskip}{0.7\baselineskip} Метод доступа & \centrow Тип данных & \centrow Описание \\
\hline \centrow 1 & \centrow 2 & \centrow 3 & \centrow 4\\ \hline
\endfirsthead
\caption*{Продолжение таблицы \ref{classdataprocessing:table}}\\
\hline \centrow 1 & \centrow 2 & \centrow 3 & \centrow 4\\ \hline
\finishhead
tree & public & xml.etree.ElementTree & Объект дерева XML, содержащий структурированные данные из файла XML. Значением является адрес файла для информирования. \\ 
\hline root & public & xml.etree.ElementTree & Корневой элемент XML файла, предоставляющий доступ к дочерним элементам. Парсинг.\\ 
\hline num\_objects & public & int & Количество объектов, размеченных в XML файле. Количество размеченных объектов (6 - кол-во служебных элементов, таких как размер, название и т.д)\\ 
\hline load\_img & public & Function & Функция для загрузки и предобработки изображения с использованием TensorFlow. \\ 
\hline cords & public & list & Список для хранения нормализованных координат объектов. \\ 
\hline w & public & int & Ширина изображения, извлеченная из XML файла. \\ 
\hline h & public & int & Высота изображения, извлеченная из XML файла. \\ 
\hline object\_cords & public & list & Список для хранения нормализованных координат одного объекта. Тут мы также нормализуем координаты от -1 до 1, опираясь на исходные координаты. \\ 
\hline img & public & Tensor & Тензор изображения после загрузки и предобработки. \\ 
\hline plt.figure & public & Function & Функция для создания новой фигуры в Matplotlib. \\ 
\hline plt.subplot & public & Function & Функция для добавления подграфика в текущую фигуру. \\ 
\hline plt.imshow & public & Function & Функция для отображения изображения в подграфике.  \\ 
\hline plt.axis & public & Function & Функция для управления отображением осей графика. \\ 
\hline plt.show & public & Function & Функция для отображения всей фигуры с подграфиками. Открывает окно с исходным изображением. \\
\end{xltabular}
\renewcommand{\arraystretch}{1.0} % восстановление сетки

\subsubsection{Класс creatingtfrecordclassifier}

Класс creatingtfrecordclassifier отвечает за создание и загрузку изображений и создания их записи в формате .tfrecord. В этом классе подготавливаются данные для дальнейшего использования в других классах и главное - создание файла classifier\_dataset.tfrecord . 
Описание полей и методов данного класса представлено в таблице \ref{classtfrecordclassifier:table}.

\renewcommand{\arraystretch}{0.8} % уменьшение расстояний до сетки таблицы
\begin{xltabular}{\textwidth}{|X|p{1.5cm}|>{\setlength{\baselineskip}{0.7\baselineskip}}p{2.85cm}|>{\setlength{\baselineskip}{0.7\baselineskip}}p{3.85cm}|}
\caption{Спецификация полей класса <<creatingtfrecordclassifie>> \label{classtfrecordclassifier:table}}\\
\hline \centrow \setlength{\baselineskip}{0.7\baselineskip} Наименование& \centrow \setlength{\baselineskip}{0.7\baselineskip} Метод доступа & \centrow Тип данных & \centrow Описание \\
\hline \centrow 1 & \centrow 2 & \centrow 3 & \centrow 4\\ \hline 
\endfirsthead
\caption*{Продолжение таблицы \ref{classtfrecordclassifier:table}}\\
\hline \centrow 1 & \centrow 2 & \centrow 3 & \centrow 4\\ \hline
\finishhead
fn & public & str & Путь к папке с изображениями. Значением является адрес папки с нашими изображениями и xml файлами.\\ 
\hline check\_xml\_list & public & function & Функция для создания списка XML файлов в указанной папке. Формируем список всех xml файлов в папке.\\ 
\hline load\_img & public & function & Функция для загрузки и предобработки изображения. \\ 
\hline create\_load\_tfrec\_for\_classifier & public & function & Функция для создания TFRecord для классификатора. Здесь мы преобразуем папку в tfrecord для классификарора\\ 
\hline namespace & public & dictionary & Словарь для сопоставления названий с метками классов. В нашем случае 0 - Ничего (nothing), 1- Возгорание (fire). \\ 
\hline saveinrecord & public & function & Внутренняя функция для сохранения обработанного изображения и метки в TFRecord. Создается запись writer, данные предоставляются в байтовом виде и собирается экземляр, который потом записывается в запись writer.\\ 
\hline parse\_record & public & function & Функция для разбора записей TFRecord. Имена элементов остаются как при записи
\end{xltabular}
\renewcommand{\arraystretch}{1.0} % восстановление сетки

\subsubsection{Класс creatingtfrecordlocalizer}

Класс creatingtfrecordlocalizer отвечает за создание и загрузку изображений и создания их записи в формате .tfrecord. В этом классе подготавливаются данные для дальнейшего использования в других классах и главное - создание файла localizer\_dataset.tfrecord . 
Описание полей и методов данного класса представлено в таблице \ref{classtfrecordlocalizer:table}.

\renewcommand{\arraystretch}{0.8} % уменьшение расстояний до сетки таблицы
\begin{xltabular}{\textwidth}{|X|p{1.5cm}|>{\setlength{\baselineskip}{0.7\baselineskip}}p{3cm}|>{\setlength{\baselineskip}{0.7\baselineskip}}p{3.85cm}|}
\caption{Спецификация полей класса <<creatingtfrecordlocalizer>> \label{classtfrecordlocalizer:table}}\\
\hline \centrow \setlength{\baselineskip}{0.7\baselineskip} Наименование& \centrow \setlength{\baselineskip}{0.7\baselineskip} Метод доступа & \centrow Тип данных & \centrow Описание \\
\hline \centrow 1 & \centrow 2 & \centrow 3 & \centrow 4\\ \hline
\endfirsthead
\caption*{Продолжение таблицы \ref{classtfrecordlocalizer:table}}\\
\hline \centrow 1 & \centrow 2 & \centrow 3 & \centrow 4\\ \hline
\finishhead
fn & public& string & Путь к папке с изображениями. Значением является адрес папки с нашими изображениями и xml файлами. \\ 
\hline load\_img & private & function & Загружает изображение, декодирует, нормализует и изменяет размер. \\ 
\hline create\_load\_tfrec\_for\_localizer & private & function & Создает TFRecord для локализатора из XML файлов. \\ 
\hline p & private & list & Формируем список всех xml файлов в папке. \\ 
\hline writer & private & TFRecordWriter & Записывает данные в TFRecord. Создание самой записи \\ 
\hline tree & private & ElementTree & Адрес файла \\ 
\hline root & private & Element & Парсит XML файлы. \\ 
\hline num\_objects & private & int & Количество объектов в XML файле. \\ 
\hline cords & private & list & Список координат объектов. \\ 
\hline w & private & int & Ширина изображения. \\
 \hline h & private & int & Высота изображения. \\ 
 \hline object\_cords & private & list & Координаты одного объекта. Тут мы также нормализуем координаты от -1 до 1, опираясь на исходные координаты. \\ 
 \hline img & private & Tensor & Тензор изображения. \\ 
 \hline serialized\_img & private & bytes & Сериализованное изображение. Готовим данные, представляем в байтовом виде.\\ 
 \hline serialized\_cords & private & bytes & Сериализованные координаты. Готовим данные, представляем в байтовом виде.\\ 
 \hline example & private & Example & Пример данных для TFRecord. Собираем экзепмляр\\ 
 \hline dataset & private & TFRecordDataset & Набор данных из TFRecord. Сразу после создания проверка чтения записи. \\ 
 \hline parse\_record & private & function & Разбирает запись из TFRecord. Имена элементов как при записи\\ 
 \hline feature\_description & private & dict & Описание приходящего экземпляра. \\ 
 \hline parsed\_record & private & dict & Разобранный экземпляр. 
\end{xltabular}
\renewcommand{\arraystretch}{1.0} % восстановление сетки

\subsubsection{Класс classifier}

Класс classifier отвечает за работу с нейронной сетью по классификации данных, её обучение и тестирование. В этом классе  реализован парсинг элементов из tfrecord, загрузка и создание модели нейросети, а также построение её архитектуры.
Описание полей и методов данного класса представлено в таблице \ref{classifier:table}.

\renewcommand{\arraystretch}{0.8} % уменьшение расстояний до сетки таблицы
\begin{xltabular}{\textwidth}{|X|p{1.5cm}|>{\setlength{\baselineskip}{0.7\baselineskip}}p{2.85cm}|>{\setlength{\baselineskip}{0.7\baselineskip}}p{4.85cm}|}
\caption{Спецификация полей класса <<classifier>> \label{classifier:table}}\\
\hline \centrow \setlength{\baselineskip}{0.7\baselineskip} Наименование& \centrow \setlength{\baselineskip}{0.7\baselineskip} Метод доступа & \centrow Тип данных & \centrow Описание \\
\hline \centrow 1 & \centrow 2 & \centrow 3 & \centrow 4\\ \hline
\endfirsthead
\caption*{Продолжение таблицы \ref{classifier:table}}\\
\hline \centrow 1 & \centrow 2 & \centrow 3 & \centrow 4\\ \hline
\finishhead
\hline parse\_record & public & function & Разбирает запись TFRecord, возвращая изображение и имя. \\ 
\hline shuffle, cache, prefetch, batch & public & method & Подготавливает датасет к обучению: перемешивание, кэширование, предварительная загрузка, батчинг. \\ 
\hline test\_classifier & public & function & Визуализирует примеры из датасета и их метки. \\ 
\hline Model & public & class & Определяет модель нейросети с методами для обучения. \\ 
\hline training\_step & public & method & Выполняет шаг обучения, возвращая среднее значение потерь. \\ 
\hline trainclass & public & function & Обучает классификатор и визуализирует диаграмму потерь. \\ 
\hline imshow\_and\_pred & public & function & Визуализирует изображения и предсказания модели с помощью matplotlib. \\ 
\hline saveclassifier & public & function & Сохраняет обученную модель классификатора.
\end{xltabular}
\renewcommand{\arraystretch}{1.0} % восстановление сетки


\subsubsection{Класс training}

Класс training отвечает за работу с нейронной сетью по локализации данных, её обучение и тестирование. В этом классе  реализован парсинг элементов из tfrecord, загрузка и создание модели нейросети, а также построение её архитектуры и явное обучение.
Описание полей и методов данного класса представлено в таблице \ref{localizer:table}.

\renewcommand{\arraystretch}{0.8} % уменьшение расстояний до сетки таблицы
\begin{xltabular}{\textwidth}{|X|p{1.5cm}|>{\setlength{\baselineskip}{0.7\baselineskip}}p{2.85cm}|>{\setlength{\baselineskip}{0.7\baselineskip}}p{4.85cm}|}
\caption{Спецификация полей класса <<training>> \label{localizer:table}}\\
\hline \centrow \setlength{\baselineskip}{0.7\baselineskip} Наименование& \centrow \setlength{\baselineskip}{0.7\baselineskip} Метод доступа & \centrow Тип данных & \centrow Описание \\
\hline \centrow 1 & \centrow 2 & \centrow 3 & \centrow 4\\ \hline
\endfirsthead
\caption*{Продолжение таблицы \ref{localizer:table}}\\
\hline \centrow 1 & \centrow 2 & \centrow 3 & \centrow 4\\ \hline
\finishhead
parse\_record & public & function & Разбирает запись TFRecord, возвращая изображение и координаты. \\
\hline IoU\_Loss & public & function & Вычисляет потери IoU между истинными и предсказанными рамками. \\
\hline Model & public & class & Определяет модель нейросети с методами для обучения. \\
\hline training\_step & public & method & Выполняет шаг обучения, возвращая значение потерь. \\
\hline savemodel & public & function & Сохраняет обученную модель. \\
\hline loadmodel & public & function & Загружает веса модели. \\
\hline testing & public & function & Тестирует модель, визуализируя результаты. \\
\hline train & public & function & Обучает модель и визуализирует историю потерь.
\end{xltabular}
\renewcommand{\arraystretch}{1.0} % восстановление сетки


\subsection{Модульное тестирование разработанного приложения}

Модульные тесты для класса main из модели данных представлены на рисунках \ref{test1:image}-\ref{test3:image}.

\begin{figure}[H]
\begin{lstlisting}[language=Python]
import unittest
from main import detect_objects, visualize, prettify

class TestFireDetection(unittest.TestCase):

    def setUp(self):
        self.test_image = np.zeros((1024, 1024, 3), dtype=np.uint8)
        self.test_cords = np.array([[10, 20, 30, 40] for _ in range(10)])
        self.test_classes = np.array([1 for _ in range(10)])
        self.test_probs = np.array([0.9 for _ in range(10)])

    def test_detect_objects(self):
        cords, classes, probs = detect_objects(self.test_image)
        self.assertEqual(len(cords), 10)
        self.assertEqual(len(classes), 10)
        self.assertEqual(len(probs), 10)
        self.assertTrue((cords >= 0).all() and (cords <= 1024).all())
        self.assertTrue((classes >= 0).all() and (classes <= 1).all())
        self.assertTrue((probs >= 0).all() and (probs <= 1).all())

    def test_visualize(self):
        result_image = visualize(self.test_image, self.test_cords, self.test_classes, self.test_probs)
        self.assertIsNotNone(result_image)
        self.assertEqual(result_image.shape, self.test_image.shape)

    def test_prettify(self):
        new_cords, new_classes, new_probs = prettify(self.test_cords, self.test_classes, self.test_probs)
        self.assertIsNotNone(new_cords)
        self.assertIsNotNone(new_classes)
        self.assertIsNotNone(new_probs)
        self.assertTrue(len(new_cords) <= len(self.test_cords))
\end{lstlisting}  
\caption{Модульный тест основных методов распознавания, визуализации и объединения класса main}
\label{test1:image}
\end{figure}

\begin{figure}[H]
\begin{lstlisting}[language=Python]
import unittest
from unittest.mock import patch
from main import detect_fire_in_image, loadimage, detect

class TestFireDetection(unittest.TestCase):

    def setUp(self):
        self.test_path = 'D:/NeuroPractice/NeuroPractice/src/images/fire1.jpg'
        self.test_image = np.zeros((1024, 1024, 3), dtype=np.uint8)
        
     def test_detect_fire_in_image(self):
        result = detect_fire_in_image(self.test_path)
        self.assertIsNotNone(result)
        self.assertEqual(result.shape, (1024, 1024, 3))
    def test_loadimage(self, mock_open):
        mock_open.return_value.__enter__.return_value = 'fake file content'
        image = loadimage('fake/path/to/image.jpg')
        mock_open.assert_called_with('fake/path/to/image.jpg', 'rb')
        self.assertIsNotNone(image)

    @patch('main.detect')
    def test_detect(self, mock_detect):
        mock_detect.return_value = True
        result = detect('fake/path/to/image.jpg')
        mock_detect.assert_called_once()
        self.assertTrue(result)
\end{lstlisting}  
\caption{Модульный тест методов распознавания возгораний, загрузки изображения и вложенного метода detect класса main}
\label{test2:image}
\end{figure}

\begin{figure}[H]
\begin{lstlisting}[language=Python]
import unittest
from your_module import detect_objects
import tensorflow as tf
import numpy as np

class TestObjectDetection(unittest.TestCase):
    def test_detect_objects(self):
        test_image = np.random.randint(0, 256, (128, 128, 3), dtype=np.uint8)
        test_image = tf.convert_to_tensor(test_image, dtype=tf.float32)

        cords, classes, probs = detect_objects(test_image)

        self.assertIsInstance(cords, np.ndarray, "Координаты должны быть массивом NumPy")
        self.assertIsInstance(classes, np.ndarray, "Классы должны быть массивом NumPy")
        self.assertIsInstance(probs, list, "Вероятности должны быть списком")

        self.assertEqual(cords.shape, (10, 4), "Массив координат должен иметь форму (10, 4)")
        self.assertEqual(len(classes), 10, "Массив классов должен содержать 10 элементов")
        self.assertEqual(len(probs), 10, "Список вероятностей должен содержать 10 элементов")

        for prob in probs:
            self.assertGreaterEqual(prob, 0, "Вероятность должна быть больше или равна 0")
            self.assertLessEqual(prob, 1, "Вероятность должна быть меньше или равна 1")
\end{lstlisting}  
\caption{Отдельный модульный тест метода распознавания класса main}
\label{test3:image}
\end{figure}



\subsection{Системное тестирование разработанного приложения}
В целях проверки работоспособности программно-информационной системы было проведено системное тестирование. Описание тестов, их результаты и скриншоты экрана представлены в данном разделе.

На рисунке \ref{main:image} представлена главная страница сайта «Русатом – Аддитивные технологии».
\newpage % при необходимости можно переносить рисунок на новую страницу
\begin{figure}[H] % H - рисунок обязательно здесь, или переносится, оставляя пустоту
\center{\includegraphics[width=1\linewidth]{main1}}
\center{\includegraphics[width=1\linewidth]{main2}}
\center{\includegraphics[width=1\linewidth]{main3}}
\caption{Главная страница сайта «Русатом – Аддитивные технологии»}
\label{main:image}
\end{figure}

На рисунке \ref{systemtest1:image} представлен интерфейс программы.
\begin{figure}[H]
\centering
\includegraphics[width=1\linewidth]{systemtest1}
\caption{Интерфейс программы}
\label{systemtest1:image}
\end{figure}

На рисунках \ref{systemtest_responce:image}-\ref{systemtest_responce4:image} представлен полный путь распознания объекта на изображении.

\begin{figure}[H]
\center{\includegraphics[width=1\linewidth]{systemtest_responce}}
\caption{Диалоговое окно загрузки файла Машина.png}
\label{systemtest_responce:image}
\end{figure}

\begin{figure}[H]
\center{\includegraphics[width=1\linewidth]{systemtest_responce1}}
\caption{Изображение отображено в интерфейсе программы}
\label{systemtest_responce1:image}
\end{figure}

\begin{figure}[H]
\center{\includegraphics[width=1\linewidth]{systemtest_responce2}}
\caption{Окно выбора модели для распознания}
\label{systemtest_responce2:image}
\end{figure}

\begin{figure}[H]
\center{\includegraphics[width=1\linewidth]{systemtest_responce3}}
\caption{Распознанный объект отображен на интерфейсе}
\label{systemtest_responce3:image}
\end{figure}

\begin{figure}[H]
\center{\includegraphics[width=1\linewidth]{systemtest_responce4}}
\caption{Диалоговое окно сохранения результата}
\label{systemtest_responce4:image}
\end{figure}

На рисунках \ref{systemtest_train1:image}-\ref{systemtest_train3:image} представлены все варианты обучения нейронной сети.

На рисунке \ref{systemtest_train1:image} была нажата кнопка обучать, после чего открылось окно выбора набора.

\begin{figure}[H]
\center{\includegraphics[width=1\linewidth]{systemtest_train1}}
\caption{Окно выбора обучающего набора}
\label{systemtest_train1:image}
\end{figure}

После выбора нового набора появилось окно, предлагающее назвать модель и обучить её.

\begin{figure}[H]
\center{\includegraphics[width=1\linewidth]{systemtest_train2}}
\caption{Окно выбора названия новой модели}
\label{systemtest_train2:image}
\end{figure}

Вместо создания нового набора, после нажатия кнопки обучать, можно выбрать заготовленный набор и обучить нейронную сеть по нему.

\begin{figure}[H]
\center{\includegraphics[width=1\linewidth]{systemtest_train3}}
\caption{Выбор обучения по набору ''Белый''}
\label{systemtest_train3:image}
\end{figure}

На рисунке \ref{systemtest_responce5:image} было загружено изображение кошки.

\begin{figure}[H]
\center{\includegraphics[width=1\linewidth]{systemtest_responce5}}
\caption{Интерфейс с изображением кошки}
\label{systemtest_responce5:image}
\end{figure}

На рисунке \ref{systemtest_responce6:image} изображение кошки было обработано по модели ''Белый''.

\begin{figure}[H]
\center{\includegraphics[width=1\linewidth]{systemtest_responce6}}
\caption{Распознанныйе объекты на изображении кошки}
\label{systemtest_responce6:image}
\end{figure}
На рисунке \ref{systemtest_responce7:image} было загружено изображение лилии.

\begin{figure}[H]
\center{\includegraphics[width=1\linewidth]{systemtest_responce7}}
\caption{Интерфейс с изображением лилии}
\label{systemtest_responce7:image}
\end{figure}

На рисунке \ref{systemtest_responce8:image} изображение лилии было обработано по модели ''Белый''.

\begin{figure}[H]
\center{\includegraphics[width=1\linewidth]{systemtest_responce8}}
\caption{Распознанныйе объекты на изображении лилии}
\label{systemtest_responce8:image}
\end{figure}
На рисунке \ref{systemtest_responce9:image} было загружено изображение леса.

\begin{figure}[H]
\center{\includegraphics[width=1\linewidth]{systemtest_responce9}}
\caption{Интерфейс с изображением леса}
\label{systemtest_responce9:image}
\end{figure}

На рисунке \ref{systemtest_responce10:image} изображение леса было обработано по модели ''Белый''.

\begin{figure}[H]
\center{\includegraphics[width=1\linewidth]{systemtest_responce10}}
\caption{Распознанныйе объекты на изображении леса}
\label{systemtest_responce10:image}
\end{figure}
На рисунке \ref{systemtest_responce11:image} было загружено изображение футбола.

\begin{figure}[H]
\center{\includegraphics[width=1\linewidth]{systemtest_responce11}}
\caption{Интерфейс с изображением футбола}
\label{systemtest_responce11:image}
\end{figure}

На рисунке \ref{systemtest_responce12:image} изображение футбола было обработано по модели ''Белый''.

\begin{figure}[H]
\center{\includegraphics[width=1\linewidth]{systemtest_responce12}}
\caption{Распознанныйе объекты на изображении футбола}
\label{systemtest_responce12:image}
\end{figure}
На рисунке \ref{systemtest_responce13:image} было загружено изображение платья.

\begin{figure}[H]
\center{\includegraphics[width=1\linewidth]{systemtest_responce13}}
\caption{Интерфейс с изображением платья}
\label{systemtest_responce13:image}
\end{figure}

На рисунке \ref{systemtest_responce14:image} изображение платья было обработано по модели ''Белый''.

\begin{figure}[H]
\center{\includegraphics[width=1\linewidth]{systemtest_responce14}}
\caption{Распознанныйе объекты на изображении платья}
\label{systemtest_responce14:image}
\end{figure}

На рисунке \ref{systemtest_responce15:image} было загружено изображение берёзы.

\begin{figure}[H]
\center{\includegraphics[width=1\linewidth]{systemtest_responce15}}
\caption{Интерфейс с изображением берёзы}
\label{systemtest_responce15:image}
\end{figure}

На рисунке \ref{systemtest_responce16:image} изображение берёзы было обработано по модели ''Белый''.

\begin{figure}[H]
\center{\includegraphics[width=1\linewidth]{systemtest_responce16}}
\caption{Распознанныйе объекты на изображении берёзы}
\label{systemtest_responce16:image}
\end{figure}

На рисунке \ref{systemtest_responce17:image} было загружено изображение дварфа.

\begin{figure}[H]
\center{\includegraphics[width=1\linewidth]{systemtest_responce17}}
\caption{Интерфейс с изображением дварфа}
\label{systemtest_responce17:image}
\end{figure}

На рисунке \ref{systemtest_responce18:image} изображение дварфа было обработано по модели ''Белый''.

\begin{figure}[H]
\center{\includegraphics[width=1\linewidth]{systemtest_responce18}}
\caption{Распознанныйе объекты на изображении дварфа}
\label{systemtest_responce18:image}
\end{figure}
На рисунке \ref{systemtest_responce19:image} было загружено изображение Чебурашки.

\begin{figure}[H]
\center{\includegraphics[width=1\linewidth]{systemtest_responce19}}
\caption{Интерфейс с изображением Чебурашки}
\label{systemtest_responce19:image}
\end{figure}

На рисунке \ref{systemtest_responce20:image} изображение Чебурашки было обработано по модели ''Белый''.

\begin{figure}[H]
\center{\includegraphics[width=1\linewidth]{systemtest_responce20}}
\caption{Распознанныйе объекты на изображении Чебурашки}
\label{systemtest_responce20:image}
\end{figure}

На рисунке \ref{systemtest_responce21:image} было загружено изображение поезда.

\begin{figure}[H]
\center{\includegraphics[width=1\linewidth]{systemtest_responce21}}
\caption{Интерфейс с изображением поезда}
\label{systemtest_responce21:image}
\end{figure}

На рисунке \ref{systemtest_responce22:image} изображение поезда было обработано по модели ''Белый''.

\begin{figure}[H]
\center{\includegraphics[width=1\linewidth]{systemtest_responce22}}
\caption{Распознанныйе объекты на изображении поезда}
\label{systemtest_responce22:image}
\end{figure}

На рисунке \ref{systemtest_responce23:image} было загружено изображение мафинов.

\begin{figure}[H]
\center{\includegraphics[width=1\linewidth]{systemtest_responce23}}
\caption{Интерфейс с изображением мафинов}
\label{systemtest_responce23:image}
\end{figure}

На рисунке \ref{systemtest_responce24:image} изображение мафинов было обработано по модели ''Белый''.

\begin{figure}[H]
\center{\includegraphics[width=1\linewidth]{systemtest_responce24}}
\caption{Распознанныйе объекты на изображении мафинов}
\label{systemtest_responce24:image}
\end{figure}
